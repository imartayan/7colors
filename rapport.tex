\documentclass[12pt,L,fira-sans]{paper}
\usepackage{code-private}

\title{Projet 7 colors}
\author{Igor \maj{Martayan} et Clément \maj{Morand}}

\begin{document}
\maketitle

\section*{Introduction}

Le but de ce projet est d'implémenter le jeu des 7 couleurs en C, avec un code aussi clair et structuré que possible.
Nous avons essayé de faire un code suffisamment générique pour qu'il soit facilement ajustable et réutilisable par la suite : cela passe par l'utilisation de structures et de méthodes générales, ainsi qu'un découpage en modules relativement indépendants les uns des autres.
Au-delà du développement du jeu en tant que tel, nous nous sommes aussi beaucoup intéressés à la conception de joueurs artificiels avec différentes stratégies : aléatoire, gloutonne, expansionniste, minimax\etc

\subsection*{Structure du projet}

Au fil des questions, nous ferons référence à plusieurs fonctions réparties dans différents modules, c'est pourquoi il peut être utile de d'expliquer rapidement la structure générale du projet pour s'y retrouver plus facilement :

\begin{itemize}
	\item \verb|structures.c| contains all the structures used in the projects : player, strategy, linked list, queue...
	\item \verb|utils.c| contains basic functions such as reading or writing a cell, choosing a random color...
	\item \verb|display.c| contains functions for printing the board, the score and the results of a game
	\item \verb|input.c| contains functions for asking game modes, next move and new game
	\item \verb|board.c| contains functions for creating the board and updating it after each move
	\item \verb|strategies.c| contains different strategies for artificial intelligence
	\item \verb|game.c| contains functions for selecting strategies and running normal or fast games
	\item \verb|main.c| contains the main loop of the game
\end{itemize}

\section{Voir le monde en 7 couleurs}

\begin{qu}
	(fonction \verb|rand| de \verb|stdlib|, joueurs représentés par \verb|'1'| et \verb|'2'|)
\end{qu}

\begin{qu}
	("serpent", nombre de parcours est en \(\Theta(n^2)\).)
\end{qu}

\begin{qu}
	(parcours en largeur, file)
\end{qu}

\section{À la conquête du monde}

\begin{qu}
	(limites d'implémentation ?)
\end{qu}

\begin{qu}
	(voir \verb|game_ended|)
\end{qu}

\section{La stratégie de l'aléa}

\begin{qu}
	(rien de passionnant)
\end{qu}

\begin{qu}
	(toujours un bfs)
\end{qu}

\section{La loi du plus fort}

\begin{qu}
	(plus intéressant, on simule chaque coup possible)
\end{qu}

\begin{qu}
	(glouton ftw, équitable ? terrain symétrique ? premier joueur avantagé)
\end{qu}

\begin{qu}
	(glouton ftw)
\end{qu}

\section{Les nombreuses huitièmes merveilles du monde}

\begin{qu}
	(glouton < hégémonique)
\end{qu}

\begin{qu}
	(cf minimax, \(7^n\) ?)
\end{qu}

\section{Le pire du monde merveilleux des 7 couleurs}

\begin{qu}
	(compliqué)
\end{qu}

\begin{qu}
	(cf hégémonique ?)
\end{qu}

\section*{Synthèse}

\section*{Bibliographie}

\begin{itemize}
	\item \url{http://www.gecif.net/qcm/information/ascii_decimal_hexa.pdf}
	\item \url{http://web.theurbanpenguin.com/adding-color-to-your-output-from-c/}
	\item \url{https://en.wikipedia.org/wiki/Minimax}
	\item \url{https://en.wikipedia.org/wiki/Alpha%E2%80%93beta_pruning}
\end{itemize}

\end{document}
